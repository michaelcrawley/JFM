%  The dissertation abstract can only be 500 words.
It has been well-known within the aeroacoustic community that the dominant noise sources in high-speed turbulent jets are related to the evolution of the large-scale structures as they convect downstream.
However, the exact dynamics of these large-scale structures which are relevant to the noise generation process are less clear.
This work aims to study the dynamics of, and noise generated by, the large-scale structures in high-fidelity in a Mach 0.9 turbulent jet using simultaneous pressure and velocity data acquisition systems alongside plasma-based excitation to produce either individual or periodic coherent ring vortices in the shear layer.

In the first phase, the irrotational near-field pressure is decomposed into its constitutive acoustic and hydrodynamic components, and two-point cross-correlations are used between the acoustic near-field and far-field in order to identify the dominant noise source region.
The large-scale structure interactions were then investigated by stochastically-estimating the time-resolved velocity fields from time-resolved near-field pressure traces and non-time-resolved planar velocity snapshots using an artificial neural network.
The coherent structures generated by the excitation were then identified and tracked using standard vortex identification routines.
Finally, from Ribner's dilatation-based acoustic analogy the aeroacoustic source terms were computed using the time-resolved velocity field produced by the stochastic estimation.
The results indicated that the disintegration of the coherent ring vortices are the dominant aeroacoustic source mechanism for the Mach 0.9, high Reynolds number jet studied here. 
However, the merging of vortices in the initial shear layer was also identified as a non-trivial noise source mechanism in high-speed, turbulent jets. 