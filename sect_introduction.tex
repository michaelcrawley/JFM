\section{Introduction}
\label{introduction}
The advent of the turbojet engine led to a transformation in both commercial and military aviation, allowing for much faster flight than previously possible with propeller-driven aircraft. 
However, the increased thrust of turbojets (and turbofans more recently) has come at great cost; significant acoustic radiation is generated by the rotating components (compressor, turbine, fan), by the combustion process, and ultimately by the free jet itself. 
This has spurred extensive research into the acoustic source mechanism in high speed, high Reynolds number jets.
While progress has been made in the field of aeroacoustics, understanding of jet noise sources and their radiation mechanisms remains incomplete \citep{Jordan2008}.
This is due to the large number of interrelated parameters (e.g. Reynolds number, temperature ratio, acoustic Mach number, nozzle geometry, et cetera) as well as the large disparity in the associated length and time scales of the turbulent phenomena and the radiated noise.
As a result, current noise-mitigation technologies for free jets have largely been applied in an ad
hoc manner, due to the community's incomplete understanding of the aeroacoustic sources.
Fully realizing this maximum noise reduction potential will require a much more detailed understanding of the mechanism (or mechanisms) by which free jets radiate to the far-field.

It is generally agreed that the dominant noise sources in high-speed jets are related to the large-scale coherent structures which have been identified in the jet mixing layer \citep{Arndt1997}. 
As discussed by \citet{Tam1995} (among others), large-scale structures can be represented as instability waves superimposed upon the mean flow.
At subsonic convection velocities, a plane instability wave with fixed frequency-wavenumber will emit no acoustic radiation to the far-field.
However, modulation of the instability wave's amplitude creates a dispersion in the energy content of the instability wave.
By doing so, the broadband instability wave, commonly referred to as a \emph{wavepacket}, can shift energy to supersonic phase-velocities and hence produce sound which radiates to the far-field.
Wavepacket models for noise emission have become commonplace, owing to their great success at predicting low-angle acoustic emission \citep{Obrist2011}.
Simple linear wavepacket models have allowed researchers to probe different aspects of the waveform modulation, in turn illuminating possible relevant dynamical behavior of the large-scale structures for the noise generation process.
Temporal modulation of the wavepacket's amplitude and spatial extent (`jittering') were shown to increase the efficiency of the noise source \citep{Cavalieri2010}; this conforms to experimental results which have indicated that the noise generation in free jets is highly intermittent \citep{Hileman2005,Kearney-Fischer2013}. 
Though progress has been made in experimentally measuring wavepacket characteristics in high-speed turbulent jets \citep{Cavalieri2013,Baqui2014}, a direct link between large-scale structure dynamics and the aeroacoustic source has remained elusive.

In tandem to instability analysis of the noise source mechanism, the development and refinement of acoustic analogies has occurred. 
\citet{Lighthill1952} was the first to reorganize the Navier-Stokes equations into a linear wave equation with a quadrupole-like source term comprising density, Reynolds stress, and entropic fluctuations.
Work by \citet{Phillips1960}, \citet{Lilley1974}, and much later \citet{Goldstein2003} extended this formulation to include convective effects of the ambient medium, thereby separating the true sound sources in Lighthill's acoustic analogy from purely propagative effects.
Competing variations were also developed by \citet{Powell1964}, \citet{Howe1975}, and \citet{Ribner1962} (the last of which will be used in the current work) which sought to better connect the acoustic source mechanisms to distinct physical phenomena occurring in the turbulent shear layer.
In a similar vein, \citet{Cabana2008} decomposed Lighthill's acoustic source term into sub-terms of velocity, vorticity, dilatation and density fields in order to understand the relative (and sometimes competing) roles each had the the noise generation process.
Ultimately however, a clear distinction between efficiently and non-efficiently radiating structures has not been defined.

The purpose of this work is to examine the dynamical evolutions of the large-scale coherent structures which lead to the dominant mixing noise in the subsonic, turbulent jet.
The focus will be on the axisymmetric, toroidal vortices (azimuthal Fourier mode zero), as these are known to dominant the acoustic far-field.
The jet instabilities will therefore be excited using localized arc-filament plasma actuators in order to generate these axisymmetric structures with a well-defined temporal frequency and phase.
The irrotational near-field pressure will be acquired via a linear array of microphones, and decomposed into its constitutive acoustic and hydrodynamic components to identify the dominant noise source region.
Time-resolved velocity fields will then be estimated based on stochastic correlations between the near-field pressure and the orthogonal modes of ensemble (non-time-resolved) velocity field, identified by an artificial neural network.
From these time resolved fields, the coherent structure dynamics will be identified.
Lastly, the aeroacoustic source field will be computed using Ribner's Dilatation acoustic analogy \citep{Ribner1962}.
In doing so, the structure dynamics will be directly linked first to the acoustic source region, and finally to the acoustic sources themselves.