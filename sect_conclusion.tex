\section{Conclusions}
The aeroacoustic mechanisms in high-speed, turbulent jets were investigated using simultaneous pressure and velocity measurements of large-scale structures generated by plasma excitation.
As the focus of this work was on mixing noise generated by turbulent shear layer structures common to all flow regimes, rather than acoustic emission generated by supersonic flow phenomena, an unheated, Mach 0.9 jet was used.
In the current work, only structures of azimuthal mode zero (axisymmetric ring vortices) were investigated. Previous researchers have identified the axisymmetric mode as the dominant acoustic emission pattern, and this also served to simplify the data acquisition and analysis greatly by eliminating the need to obtain azimuthal velocity components and gradients.

To begin with, the irrotational near-field pressure was linearly decomposed into its constitutive hydrodynamic and acoustic components, akin to the work of previous researchers \citep{Tinney2008}.
Here though, the pressure was filtered via a two-dimensional, spatio-temporal wavelet transform, which has been found to be more robust than the Fourier transform used previously.
Once decomposed, linear correlations between the far-field acoustic signal at $30^\circ$ and the acoustic component of the near-field were computed in order to identify the dominant acoustic source region in the jet based on the measured time-lag for the peak correlation.
In all cases, natural and excited jets, the dominant acoustic source region was found to comprise the upstream region of the jet and end at $x/D \simeq 4$, just upstream of the end of the potential core.
This analysis is meant to identify only the dominant noise source region, and does not mean that no noise is generated outside of this region.
This result is in general accordance with previous results acquired at the GDTL by Hileman \etal \citep{Hileman2005}, which identified the acoustic source region using delay-and-sum beamforming with a circular microphone array in the acoustic near-field (that is, far enough such that hydrodynamic pressure effects are negligible, but not in the true geometric far-field of the jet).
In that work, the dominant acoustic source region for an unheated, Mach 1.3 jet was found to be located just downstream of the end of the potential core, and was related to the breakup of large-scale coherent structures as they passed through this region.

The evolution, interactions, and, disintegration of the large-scale structures induced by the plasma excitation were then studied by stochastically-estimating the time-resolved velocity fields from ensemble acquisition of temporally-correlated velocity snapshots and near-field pressure traces.
The velocity snapshots were first decomposed into orthogonal modes and expansion coefficients per Sirovich's method of snapshots for proper orthogonal decomposition \citep{Sirovich1987}.
A mapping from the near-field pressure to the POD expansion coefficients was generated by a standard feedforward, backpropagating neural network; from this mapping the time-resolved expansion coefficients could be estimated and thus a reduced-order, time-resolved estimate of the velocity field produced.
At very low excitation frequencies (impulsive excitation) each plasma pulse generates a single dominant structure, which initially grows rapidly as it convects downstream. 
As the structure nears the end of the potential core, $x/D \simeq 4$, a rapid disintegration of the vortex is observed, coincident with a strong axial acceleration.

As the excitation frequency is increased, a much more complex structure evolution takes form.
At moderate excitation frequencies (periodic forcing), multiple structures are initially formed by the plasma pulse; this is due to the higher-harmonics of the excitation pulse coupling with the most unstable shear layer frequency.
These initial, high-frequency structures quickly undergo a merging process (or multiple mergings) which ultimately produces large-scale coherent structures which are periodic and match the fundamental excitation frequency.
These large-scale structures later undergo a rapid disintegration and acceleration, similar to the impulsive excitation structures, as they convect downstream near the end of the potential core. 

Finally, the aeroacoustic sources were estimated from the time-resolved velocity fields using Ribner's simplified form of Lighthill's acoustic analogy, which relates fluctuations in the dilatation field to fluctuations in the acoustic field.
This required numerous simplifications to the governing equations, which ultimately degraded the accuracy of the computed aeroacoustic source field.
Unfortunately, this limited interpretation of the results, as the computed far-field acoustic signal did not match well with the measured signal.
The algorithm did reproduce fluctuating acoustic fields of similar amplitude to what was measured experimentally, but the shape of the waveforms only matched in a rough sense.
Fortunately though, broad characteristic changes in the aeroacoustic source fields with excitation could still be identified.

Analysis of the computed source fields identified that the coherent structures produced a convected wavepacket-like event, centered on the jet lipline though reaching into the potential core.
For the individual vortex rings, a clear modulation of the spatial extent and amplitude was observed just upstream of the end of the potential core.
This corresponds to the location at which the coherent ring vortex underwent a rapid disintegration as well as acceleration, and corresponds to the location at which the dominant noise events are emitted per the two-point correlations between the acoustic component of the near-field pressure and the far-field at low polar angles.
For the periodically-excited jet, an additional noise source region is observed, corresponding to the location at which the multiple smaller-scale structures undergo a consistent merging thanks to the highly consistent vortex generation by the excitation.
This secondary source became more prominent as the excitation frequency increased and the coherent vortices underwent two merging processes before decaying near the end of the potential core.

The linearity of the acoustic response of the jet (observed in the far-field) to impulsive and low-frequency periodic excitation ($St_{DF} \leq 0.25$), coupled with the observations of the vortex dynamics and acoustic source fields, indicates that the dominant source mechanism for these structures is the rapid modulation of the waveform brought on by the disintegration and acceleration of the large-scale structure as it begins to self-interact near the end of the potential core.
For the unexcited jet, in which ring vortices are highly unstable in the initial shear layer but the large-scale structures have a broad range of frequencies and phase relations to each other (and hence have less chance to merge repeatably), this is likely the dominant noise source mechanism. 
As the excitation frequency increased (at least to $St_{DF} \simeq 0.35$), the secondary source mechanism associated with the vortex pairing becomes non-negligible, which results in a modification of the far-field response as it is now a combination of these two source mechanisms.
However, given the broadband nature of the jet turbulence (in terms of both temporal and azimuthal structure), it is unlikely that the vortex merging noise source mechanism is commonly encountered in the highly turbulent jet on a regular basis.
Ultimately, this work indicates that noise suppression may be achieved by seeding the growth of higher-order azimuthal modes, thereby limiting the growth of large-scale coherent structures, which will not undergo a rapid disintegration at the end of the potential core.  